\documentclass[11pt]{article}

\begin{document}

a) The PDF of the smallest order statistic is provided by formula 6.5.4 

$g_{1}(y_{1}) = n[1-F(y_{1})]^{n-1}f(y_{1})$ 

In this case we have $g_{1}(y_{1}) = n(1-(1-e^{-y} )e^{y}$ when $y_{1} > 0$



b)The PDF of the largest order statistic is provided by formula 6.5.6
$ g_{n}(y_{n}) = n[F(y_{n})]^{n-1}f(y_{n})$ 




$ g_{n}(y_{n}) = n[1-e^{-y_{n}}]^{n-1} e^{y_{n}}$ Simplifying provides 

$ne^{-y_{n}}(1-e^{-_{n}})^{n-1}$ when $y_{n} > 0$
c) Because the exponential distribution has the memoryless property, the difference between the first order statistic and the greatest order statistic won't be conditional on the value of the first order statistic (so we can treat it as zero). The probability that all the other observations $(n-1)$ fall into the range is $P(R<r) \in (0,r)$ So $P(R<r) = 
[\int_{0}^{r} e^{-x} dx)^{n-1} = (1-e^{r})^{n-1}$ This is P(R<r) which is the CDF, differentiate to get the PDF: 
$(n-1)(1-e^{-r})^{n-2}$ $e^{-r}$ 
\end{document}

